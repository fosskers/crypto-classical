\documentclass{article}
% GENERAL
\usepackage{setspace,mathtools,amsfonts,amsmath,amsthm,amssymb,hyperref}
\usepackage{tikz,epigraph,pgfplots}  % For trees
\usepackage[utf8]{inputenc}
\usepackage{tikz-qtree,tikz-qtree-compat}
\usepackage{forest}

% FOR SOURCE CODE
\usepackage{listings}

\lstdefinelanguage{haskell}{
  morekeywords={data,type,newtype,instance,class,module,import,if,then,else,where,deriving},
  sensitive=true,
  morecomment=[l]{--},
  morestring=[b]"
}

\definecolor{dkgreen}{rgb}{0,0.6,0}
\definecolor{gray}{rgb}{0.5,0.5,0.5}
\definecolor{mauve}{rgb}{0.58,0,0.82}

% Default settings for code listings
\lstset{frame=tb,
  language=haskell,
  aboveskip=3mm,
  belowskip=3mm,
  showstringspaces=false,
  columns=flexible,
  basicstyle={\small\ttfamily},
  numbers=none,
  numberstyle=\tiny\color{gray},
  keywordstyle=\color{blue},
  commentstyle=\color{dkgreen},
  stringstyle=\color{mauve},
  frame=single,
  breaklines=true,
  breakatwhitespace=true
  tabsize=4
}

% FOR TREE DRAWING
\pgfplotsset{compat=newest}
\usetikzlibrary{shapes.geometric,arrows,fit,matrix,positioning}
\tikzset
{
      treenode/.style = {circle, draw=black, align=center, minimum size=1cm}
}

% MARGINS
\usepackage[left=1in,top=1in,right=1in,bottom=1in]{geometry}
\onehalfspacing

\begin{document}

\title{Haskell crypto-classical Library}
\author{Colin Woodbury}
\date{\today}
\maketitle

% --- TABLE OF CONTENTS ---
\tableofcontents
\clearpage
% -------------------------

\section{Introduction}
\subsection{Motivation}
The main motivation for the creation of this library was to provide
an educational tool for learners of Haskell and Cryptography. While
\href{http://hackage.haskell.org/packages/search?terms=crypto}{many}
modern Cryptographic libraries exist aleady on
\href{http://hackage.haskell.org}{Hackage} (the principal
library hosting service for Haskell), no libraries treating classical
ciphers existed. Considering many other such ``toy'' libraries are
present there, this was surprising.

Through the course of the library's implementation I took a particular
interest in Enigma. It occured to me that it may be viable
as a modern encyption scheme, should certain critical modifications
be made. These are described in detail in the Enigma section below.

\subsection{Installation}
The official release of \emph{crypto-classical} is available on Hackage
here: \url{here}

\subsubsection{Haskell Platform}
This library requires \emph{ghc} $\geq$ 7.8 and
\emph{cabal-install} $\geq$ 1.20.\\
They are available here: \url{https://www.haskell.org/downloads}

\subsubsection{Compiling}
In an empty directory within a Terminal,
run \fbox{cabal sandbox init}. This will create a
sandbox for dependency modules, so that the global state of your Haskell
packages won't be affected.\\
Next, run \fbox{cabal install crypto-classical}. This will install the
library and its dependencies into your sandbox.

\subsubsection{Testing in a REPL}
Run \fbox{cabal repl} in your sandboxed directory. Inside the repl,
execute \fbox{import Crypto.Classical} to import every function
exported by the library. We can then perform simple, manual tests:

\begin{lstlisting}
> import Crypto.Classical
> import Control.Lens
> :set -XOverloadedStrings
> (key <$> prng) >>= \k -> return (encrypt k "What a great day for an attack!" ^. enigma)
\end{lstlisting}

See the \textbf{Testing} section below for more information.

\subsection{Design}
Haskell was chosen as the implementation language as it is very
aesthetically and functionally mathematical,
which allows for short, expressive code.\\
Below are described several conveniences afforded us through Haskell.

\subsubsection{Type-level Modular Arithmetic}
The library \emph{modular-arithmetic}, alongside several built-in
optional compiler extensions (marked with \emph{LANGUAGE}),
provides us with a means for defining
modular integer Rings of any positive size.

\lstset{extendedchars=false,
escapeinside=**}
\begin{lstlisting}
{-# LANGUAGE DataKinds #-}
{-# LANGUAGE TypeOperators #-}

-- Yields 6. *$\mathbb{Z}$* is a type alias for Integer, and 26 is the period.
foo :: *$\mathbb{Z}$*/26
foo = 20 + 12

-- Yields [1,2,3,4,5,6,7,8,9,10,11,12,13,14,15,16,17,18,19,20,21,22,23,24,25]
-- This would normally be an infinite list with any other number type.
bar :: [*$\mathbb{Z}$*/26]
bar = [1..]
\end{lstlisting}

\subsubsection{Random Keys and Polymorphic Cipher Choice}
Two simple \emph{typeclasses} let us define a type-level Key-Cipher
relationship, as well as a means of polymorphically generating keys.

\begin{lstlisting}
-- | Keys can appear in a number of different forms.
-- E.g. a single number, a tuple, a mapping, etc.
-- Each needs to be interpreted uniquely by a Cipher's
-- `encrypt` and `decrypt` algorithms.
class Key a where
  -- | Randomly generate a Key.
  key :: CPRG g => g -> a

-- | A Cipher must be able to encrypt and decrypt. The Cipher type
-- determines the Key type.
class Key k => Cipher k a | a -> k where
  encrypt :: k -> ByteString -> a ByteString
  decrypt :: k -> ByteString -> a ByteString  
\end{lstlisting}

Given an instance of the \emph{Key} class for some key type,
we can generate keys with the \emph{key} function just by constricting
its return type.

\begin{lstlisting}
instance Key (*$\mathbb{Z}$*/26) where
  key g = toMod . fst $ generateBetween g 1 25
  
caesarKey :: CPRG g => g -> *$\mathbb{Z}$*/26
caesarKey = key 
\end{lstlisting}

Or, thanks to the \emph{Cipher} class defined above, we can force the key
type using our Cipher ``unwrapper'' functions. You can think of \fbox{\^{}.}
 operator as meaning ``with'' or ``as''.

\begin{lstlisting}
foo :: CPRG g => g -> ByteString
foo g = encrypt (key g) "Haskell sure is great." ^. caesar

bar :: CPRG g => g -> ByteString
bar g = encrypt (key g) "Haskell sure is great." ^. stream
\end{lstlisting}

Our unwrapper functions (\emph{caesar} and \emph{stream} here) force the
types of both \emph{encrypt} and \emph{key}.

\subsubsection{Lenses}
A relatively recent addition to the Funtional Programming world, Lenses
provide a light-weight interface to access and modify data fields in
(potentially) nested,
immutable data structures. For instance, the \emph{Enigma} module
simulates Rotor rotation with this short function using Lenses:

\begin{lstlisting}
-- | Turn the (machine's) right-most (left-most in List) Rotor by one
-- position. If its turnover value wraps back to 25, then turn the next
-- Rotor as well.
turn :: [Rotor] -> [Rotor]
turn [] = []
turn (r:rs) = if (r' ^. turnover) == 25 then r' : turn rs else r' : rs
  where r' = r & circuit %~ rotate 1 & turnover -~ 1
\end{lstlisting}

\subsection{Licensing}
\fbox{crypto-classical} is available under the terms of the \textbf{BSD3}
license.

% === %

\section{Simple Ciphers}
For more information on these Ciphers, see the documentation provided
in the source code.

\subsection{Caesar}
\textbf{Module:} Crypto.Classical.Cipher.Caesar\\
\textbf{Key Type:} A shift value.
A number in $\mathbb{Z}_{26}$, except 0. We exclude
0 to maintain a requirement of all our ciphers: No message may encrypt
to itself.\\
\textbf{Encryption:} Add the shift value to each character (as a number).
The modular type $\mathbb{Z}/26$ wraps values over or under 25 automatically.\\
\textbf{Decryption:} Flip the sign of the key and encrypt again.

\subsection{Affine}
\textbf{Module:} Crypto.Classical.Cipher.Affine\\
\textbf{Key Type:} A pair of numbers $(a,b)$, each in $\mathbb{Z}_{26}$. $a$
must be coprime with 26 or decryption is impossible.\\
\textbf{Encryption:} \fbox{$E_k(c) = ac + b \mod 26$} for every $c$ in the plaintext.\\
\textbf{Decryption:} \fbox{$D_k(c) = a^{-1} \cdot (c - b)$} for every $c$
in the ciphertext.

\subsection{Substitution}
\textbf{Module:} Crypto.Classical.Cipher.Substitution\\
\textbf{Key Type:} A random mapping (Haskell's \emph{Map} type) between
plaintext letters and cipher letters. A letter can map to itself.
Technically here the ``identity'' key also exists by this algorithm,
but the probability
of generating it is $\frac{1}{26!}$ (\~{}29 quintillion times harder than
winning the Lotto 649).\\
\textbf{Encryption:} For each plaintext character, simply look it up
in our \emph{Map}.\\
\textbf{Decryption:} Invert the Map (possible, since the Map is bijective
on capital Latin letters) and encrypt.

\subsection{Stream}
\textbf{Module:} Crypto.Classical.Cipher.Stream\\
\textbf{Key Type:} An infinite list of shift values (integers) in
$\mathbb{Z}_{26}$.\\
\textbf{Encryption:} For each letter in the plaintext, add its corresponding
shift value from the key.\\
\textbf{Decryption:} Multiply every (yes, every) value in the key by $-1$,
and encrypt. This is possible because Haskell is lazy, and will only
flip the sign of as many key values as it needs to.

\subsection{Vigenère}
\textbf{Module:} Crypto.Classical.Cipher.Vigenere\\
\textbf{Key Type:} An infinite list of shift values (integers) in
$\mathbb{Z}_{26}$ with a repeating pattern.
A key generated by \emph{key} has no pattern initially.
It borrows \emph{key} from the \emph{Stream} instance,
shortens the key generated to a factor of the plaintext length,
and then repeats
that pattern forever.\\
\textbf{Encryption:} Call the \emph{encrypt} function from \emph{Stream}
with the Vigenère key.\\
\textbf{Decryption:} Call the \emph{decrypt} function from \emph{Stream}
with the Vigenère key.

% === %

\section{Enigma}

\subsection{Design Notes}
All wiring information in this section was found at the
\href{http://www.cryptomuseum.com/crypto/enigma/wiring.htm}{Crypto Museum}
website. Much more detail can be found in the documentation within
the source code.

\subsubsection{Models}
There were many Enigma models, both commercial and military. This library
provides an implementation of the Wehrmacht Enigma I (1930-38), which
had 3 rotors. Two extra choices of rotors were added in December 1938, and
our implementation reflects this.

\subsubsection{Rotor Wirings}
Rotors are referred to by Roman numerals. They each have unique
letter mappings and turnover points.
\begin{center}
\begin{tabular}{l | c | c | l}
  Rotor \# & ABCDEFGHIJKLMNOPQRSTUVWXYZ & Turnover Position & Introduced\\
  \hline
  I & EKMFLGDQVZNTOWYHXUSPAIBRCJ & Q $\to$ R & 1930\\
  II & AJDKSIRUXBLHWTMCQGZNPYFVOE & E $\to$ F & 1930\\
  III & BDFHJLCPRTXVZNYEIWGAKMUSQO & V $\to$ W & 1930\\
  IV & ESOVPZJAYQUIRHXLNFTGKDCMWB & J $\to$ K & Dec 1938\\
  V & VZBRGITYUPSDNHLXAWMJQOFECK & Z $\to$ A & Dec 1938
\end{tabular}
\end{center}

Rotor details: \url{https://en.wikipedia.org/wiki/Enigma_rotor_details}

\subsubsection{Reflector Wirings}
There were also several reflector models. While some rotated, the army
and air force reflectors did not.\\
We use the \emph{Umkehrwalze B}, which appeared in November 1937
and would have been in use when the extra rotor choices were added in 1938.
The \emph{Umkehrwalze C} was only used briefly in 1940.\\

\begin{center}
  \begin{tabular}{l | c}
    Reflector & ABCDEFGHIJKLMNOPQRSTUVWXYZ\\
    \hline
    Umkehrwalze A & EJMZALYXVBWFCRQUONTSPIKHGD\\
    Umkehrwalze B & YRUHQSLDPXNGOKMIEBFZCWVJAT\\
    Umkehrwalze C & FVPJIAOYEDRZXWGCTKUQSBNMHL
  \end{tabular}
\end{center}

\subsection{Key}
The encryption key has five components:
\begin{enumerate}
\item Choice of three rotors from five
\item Ordering of the rotors
\item Alphabet ring position relative to rotor (doesn't contribute entropy)
\item Initial rotor positions
\item Plugboard wiring
\end{enumerate}
The implementation presented in this library choses to ignore Alphabet ring
position relative to the rotor. In fact, we largely ignore the concept of
letters altogether. A rotor is just a series of entry points mapping
to exit points, meaning we can represent these numerically
(as \emph{\_circuit} below):

\begin{lstlisting}
data Rotor = Rotor { _name     :: Text
                   , _turnover :: *$\mathbb{Z}$*/26
                   , _circuit  :: Map (*$\mathbb{Z}$*/26) (*$\mathbb{Z}$*/26) } deriving (Eq,Show)  
\end{lstlisting}

\subsection{Encryption}
The mechanical ordering of the encryption steps can be described
mathematically as:

\begin{align*}
  E &= PRMLUL^{-1}M^{-1}R^{-1}P^{-1}
\end{align*}

Where:
\begin{itemize}
\item $P$ is the plugboard mapping
\item $R$ is the right rotor
\item $M$ is the middle rotor
\item $L$ is the left rotor
\item $U$ is the reflector
\end{itemize}

The act of reflection simulates six rotors, as there was limited space
within the body of the machine. 

\subsection{A Modern Enigma Scheme}
The original Enigma was limited in how many rotors it could hold within
it. Later models squeezed a fourth one in, and some had rotating
reflectors as well. Reflectors essentially simulated a doubling of the
amount of rotors within the machine, but they also introduced the weakness
that allowed the Allies to make Known Plaintext Attacks against it.\\

How many rotors would have made Enigma (computationally)
uncrackable? Algorithms within modern computers aren't limited by physical
space as Enigma was.
Would Enigma, given $n$ sufficiently many
rotors within a computer be viable as a modern encryption tool?

\subsubsection{Modern Adaptation}
\begin{itemize}
\item \textbf{Rotors:} To be viable in a modern setting, we need to
  consider bits, not Latin letters. Thus, each rotor only has two entry
  and exit points, and
  there are only two types of rotor possible (not $26!$). These would be
  the rotor which maps a bit to itself, and that which maps a bit to
  its opposite. Let us assume that both rotor \textbf{turnover points} are
  sufficient if set to the transition between 1 and 0. This assumption may
  be false, but there are not many other positions to consider.
  \begin{center}
    Straight Rotor\\
    \begin{tabular}{c | c}
      input & output\\
      \hline
      0 & 0\\
      1 & 1
    \end{tabular}
  \end{center}
    \begin{center}
    Crossed Rotor\\
    \begin{tabular}{c | c}
      input & output\\
      \hline
      0 & 1\\
      1 & 0
    \end{tabular}
  \end{center}
    Note that traditional rotation for each rotor doesn't change the output.
    Hence, upon rotation, we will flip the wiring, making one rotor
    into the other.

\item \textbf{Plugboard:} The plugboard was essentially a programmable rotor.
 Since our modern formulation only has two rotor types possible, an analogy
 to a plugboard is not necessary.
\item \textbf{Reflector:} A reflector is also not necessary, as instead
  of reflecting back upon the previous $n$ rotors, we can just encrypt with
  $2n$ rotors and achieve a result free of the weakness of the original
  Enigma.
\item \textbf{Key:} Given $n$ rotors, we need only supply one $n$-bit
  number as the key. Since during rotation rotors become one another,
  we need only specify what they started as.\\
  Let $0$ indicate a Straight Rotor and 1 indicate a Crossed Rotor.

\end{itemize}

\subsubsection{Scheme Description}
Let $(K,E,D)$ be a triple of algorithms such that:
\begin{itemize}
\item $K$ uniformly generates a random key in $\{0,1\}^n$ where $n$ is the
  number of rotors.
\item $E_k$ encrypts one bit of the plaintext at a time by first resolving
  all rotor rotations, then sending the bit through the ``circuit'' of
  $n$ rotors.
\item $D_k(C) = E_k(C)$
\end{itemize}

\subsubsection{Key Entropy}
The Enigma as defined in this library has the following key entropy:
\begin{align*}
  \text{Rotor Choices} &= 5 \cdot 4 \cdot 3\\
  \text{Rotor Positions} &= 26 \cdot 26 \cdot 26\\
  \text{Plugboard Wiring} &= \frac{26!}{6!10!2^{10}}\\
  \text{Total Product (Keys)} &= 158,962,555,217,826,360,000 \approx 1.59 \cdot 10^{20}
\end{align*}

The number of possible keys for our modified scheme is $2^n$, where $n$
is the number of rotors. With only 68 rotors
$(2^{68} \approx 2.95 \cdot 10^{20})$ we reach a comparable number of
keys to the original. \\

\textbf{Open Question:} What is the ideal number of rotors? 1000 rotors
means keys of 1000 bits which are easy to generate,
with $2^{1000}$ possible keys.

\subsubsection{Security}
Left open for critique.

% === %

\section{Testing}
The module \fbox{Crypto.Classical.Test} provides unit tests which
generate random input data and test certain properties over all the
ciphers. We require three properties hold for the Simple Ciphers, and four
for Enigma:

\begin{enumerate}
\item \textbf{Cycle:} An encrypted plaintext should decrypt
  to the original message.
\item \textbf{No Identity:} A plaintext should never encrypt to itself.
\item \textbf{Different Keys:} Different keys should produce a
  different ciphertext given the same plaintext.
\item \textbf{(Enigma only) No Character Identity:} Any character of
  the plaintext should never encrypt to itself. This is a weakness, but is
  part of the original Enigma nonetheless.
\end{enumerate}

The function \fbox{testAll} within this module runs all the tests.

\end{document}
