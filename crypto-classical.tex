\documentclass{article}
% GENERAL
\usepackage{setspace,mathtools,amsfonts,amsmath,amsthm,amssymb,hyperref}
\usepackage{tikz,epigraph,pgfplots}  % For trees
\usepackage[utf8]{inputenc}
\usepackage{tikz-qtree,tikz-qtree-compat}
\usepackage{forest}

% FOR SOURCE CODE
\usepackage{listings}

\lstdefinelanguage{haskell}{
  morekeywords={data,type,newtype,instance,class,module,import},
  otherkeywords={=>,>>=,<-,<\%,<:,>:,\#,@},
  sensitive=true,
  morecomment=[l]{//},
  morecomment=[n]{/*}{*/},
  morestring=[b]",
  morestring=[b]',
  morestring=[b]"""
}

\definecolor{dkgreen}{rgb}{0,0.6,0}
\definecolor{gray}{rgb}{0.5,0.5,0.5}
\definecolor{mauve}{rgb}{0.58,0,0.82}

% Default settings for code listings
\lstset{frame=tb,
  language=haskell,
  aboveskip=3mm,
  belowskip=3mm,
  showstringspaces=false,
  columns=flexible,
  basicstyle={\small\ttfamily},
  numbers=none,
  numberstyle=\tiny\color{gray},
  keywordstyle=\color{blue},
  commentstyle=\color{dkgreen},
  stringstyle=\color{mauve},
  frame=single,
  breaklines=true,
  breakatwhitespace=true
  tabsize=4
}

% FOR TREE DRAWING
\pgfplotsset{compat=newest}
\usetikzlibrary{shapes.geometric,arrows,fit,matrix,positioning}
\tikzset
{
      treenode/.style = {circle, draw=black, align=center, minimum size=1cm}
}

% MARGINS
\usepackage[left=1in,top=1in,right=1in,bottom=1in]{geometry}
\onehalfspacing

\begin{document}

\title{Haskell crypto-classical Library}
\author{Colin Woodbury}
\date{\today}
\maketitle

% --- TABLE OF CONTENTS ---
\tableofcontents
\clearpage
% -------------------------

\section{Introduction}
\subsection{Motivation}
The main motivation for the creation of this library was to provide
an educational tool for learners of Haskell and Cryptography. While
\href{http://hackage.haskell.org/packages/search?terms=crypto}{many}
modern Cryptographic libraries exist aleady on
\href{http://hackage.haskell.org}{Hackage} (the principal
library hosting service for Haskell), no libraries treating classical
ciphers existed. Considering many other such ``toy'' libraries are
present there, this was surprising.

\subsection{Design}

\subsubsection{Random Keys}
\subsubsection{Type-level Modular Arithmetic}
\subsubsection{Polymorphic Cipher Choice}
\subsubsection{Lenses}
A relatively recent addition to the Funtional Programming world, Lenses
provide a means to access and modify data fields in (potentially) nested,
immutable data structures.

\subsection{Licensing}
\fbox{crypto-classical} is available under the terms of the \textbf{BSD3}
license.

% === %

\section{Simple Ciphers}

\subsection{Caesar}

\subsection{Affine}

\subsection{Substitution}

\subsection{Stream}

\subsection{Vigenère}

% === %

\section{Enigma}

\subsection{Design Notes}
All wiring information in this section was found at the
\href{http://www.cryptomuseum.com/crypto/enigma/wiring.htm}{Crypto Museum}
website.

\subsubsection{Models}
There were many Enigma models, both commercial and military. This library
provides an implementation of the Wehrmacht Enigma I (1930-38), which
had 3 rotors. Two extra choices of rotors were added in December 1938.

\subsubsection{Rotor Wirings}
Rotors are referred to by Roman numerals. They each have unique
letter mappings and turnover points.
\begin{center}
\begin{tabular}{l | c | c | l}
  Rotor \# & ABCDEFGHIJKLMNOPQRSTUVWXYZ & Turnover Position & Introduced\\
  \hline
  I & EKMFLGDQVZNTOWYHXUSPAIBRCJ & Q $\to$ R & 1930\\
  II & AJDKSIRUXBLHWTMCQGZNPYFVOE & E $\to$ F & 1930\\
  III & BDFHJLCPRTXVZNYEIWGAKMUSQO & V $\to$ W & 1930\\
  IV & ESOVPZJAYQUIRHXLNFTGKDCMWB & J $\to$ K & Dec 1938\\
  V & VZBRGITYUPSDNHLXAWMJQOFECK & Z $\to$ A & Dec 1938
\end{tabular}
\end{center}

Rotor details: \url{https://en.wikipedia.org/wiki/Enigma_rotor_details}

\subsubsection{Reflector Wirings}
There were also several reflector models. While some rotated, the army
and air force reflectors did not.\\
We should use the \emph{Umkehrwalze B}, which appeared in November 1937
and would have been in use when the extra rotor choices were added in 1938.
The \emph{Umkehrwalze C} was only used briefly in 1940.\\

\begin{center}
  \begin{tabular}{l | c}
    Reflector & ABCDEFGHIJKLMNOPQRSTUVWXYZ\\
    \hline
    Umkehrwalze A & EJMZALYXVBWFCRQUONTSPIKHGD\\
    Umkehrwalze B & YRUHQSLDPXNGOKMIEBFZCWVJAT\\
    Umkehrwalze C & FVPJIAOYEDRZXWGCTKUQSBNMHL
  \end{tabular}
\end{center}

\subsection{Key}
The encryption key has five components:
\begin{enumerate}
\item Choice of three rotors from five
\item Ordering of the rotors
\item Alphabet ring position relative to rotor (doesn't contribute entropy)
\item Initial rotor positions
\item Plugboard wiring
\end{enumerate}

\subsection{Encryption}
The mechanical ordering of the encryption steps can be described
mathematically as:

\begin{align*}
  E &= PRMLUL^{-1}M^{-1}R^{-1}P^{-1}
\end{align*}

Where:
\begin{itemize}
\item $P$ is the plugboard mapping
\item $R$ is the right rotor
\item $M$ is the middle rotor
\item $L$ is the left rotor
\item $U$ is the reflector
\end{itemize}

% === %

\section{Attacks}

\end{document}
